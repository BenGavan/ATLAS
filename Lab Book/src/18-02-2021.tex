%! Author = ben
%! Date = 22/02/2021

%%%%%%%%%%%%%% 18/02/2020 %%%%%%%%%%%%%%%%
\subsection*{\textbf{18/02/2020}}
%%%%%%%%%%%%% 9:00 %%%%%%%%%%%%%
\subsubsection*{09:00 - Lead BG}
Discussion on how to calculate cross section of a process:
\\
Integrated Luminosity = $139 \text{fb}^{-1} (\pm 1.7\%)$
\\
Coeficient $\epsilon$ given by sum of MC signal weights over that for relevent sample (found using TotExpected.Py)

\subsubsection*{09:30}
Make stacked plots for pTcone30 and etcone.

\subsubsection*{09:41}
Plot the





Add ttbar\_lep background to stack plots.


$\epsilon_{\text{Zee}}$ = 19630128.89

Quoted value for Zee cross section $76.0 \pm 0.8 \pm 2.0 \pm 2.6$ pb (https://arxiv.org/pdf/1212.4620.pdf)

%%%%%%%%%%%%% 10:38 %%%%%%%%%%%%%
\subsubsection*{10:38}

Rough estimate for the cross section: $\sigma (pp \rightarrow Z \rightarrow ee)$ is given by:
\begin{align}
    \sigma &= \frac{N^{selected} - N^{background}}{\epsilon \int L dt}
\end{align}
where
\begin{align}
    \epsilon &= \frac{\sum \text{weights for all MC events which pass selection cuts}}{\sum \text{weights for all events for that process}}
\end{align}
For $Z \rightarrow ee$ use the cuts:
\begin{itemize}
    \item lep\_n = 2
    \item same flavour/type (lep\_type [0] == lep\_type[1])
    \item opposite charge (lep\_charge [0] != lep\_charge[1])
    \item invariant mass > 60 GeV
    \subitem MC not modelled below this point
\end{itemize}
For $\sigma (pp \rightarrow Z \rightarrow ee)$:
\begin{itemize}
    \item $N^{selected} = 47531$
    \item $N^{background} = 0$
    \item $\epsilon = \frac{46740}{19630128.89} = $
    \item $\int L dt = 10.064 \textbf{fb}^{-1}$
\end{itemize}


cs = 1.9835391029941325e-09

\\
Other sources of background
- photon conversion
- hadronic jets
-W or t decays can be detected as 2 electrons or muons when one is in fact a hadron jet or electron/moun from other source.

%%%%%%%%%%%%% 14:00 %%%%%%%%%%%%%
\subsubsection*{14:00 - Lead DG}
Investigating the decay paths of W-plus and W-minus

There is an exponential decay in the number of leptons apart from a bump at ....


%%%%%%%%%%%%% 14:40 %%%%%%%%%%%%%
\subsubsection*{14:40}
Plot the invariant mass between 60-150GeV for Wplus\_2lep for events that would look like $Z \rightarrow ll$.
\\
Large underflow, so increase range to 0-150 GeV
\\
Still not totally decayed, increase range to 0-500 GeV
\\
Take the log of y-axis

%%%%%%%%%%%%% 16:10 %%%%%%%%%%%%%
\subsubsection*{16:00 - Lead BG}
Finding the background contributions from $W^+ \rightarrow l\nu_l $ with the cut
\begin{lstlisting}
     lepCut ="(" + "(lep_charge[0] != lep_charge[1]) && (lep_type[0]==11 && lep_type[1]==11) && lep_n==2 && (inv_mass_Zll > 60e3) && (inv_mass_Zll < 115e3)" + ")"
\end{lstlisting}
to select for e+e- pair.
\begin{align}
    N^{background}_{t\Bar{t}} &= 236276
    \\
    N^{background}_{W^+} &= 2247
    \\
    N^{background}_{W-} &= 1785
\end{align}
Sum of all weights for all MC events which pass cuts for Zee:
\begin{align*}
    = 4595000
\end{align*}

4817004

$\sigma(Z \rightarrow ee) = 1.94275964340403e-09 b$


TODO:
\begin{itemize}
    \item Plot $ \Delta \phi$
    \item $\epsilon$ cut
\end{itemize}



