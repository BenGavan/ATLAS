%%%%%%%%%%%%%% 09/02/2020 %%%%%%%%%%%%%%%% 
\subsection*{\textbf{09/02/2020}}
\subsection{Aims}
\begin{itemize}
    \item Establish Connection
    \item Set up Jupyter Notebook
    \item Make start on 6.x exercises
\end{itemize}

\subsubsection{Day Summary}
\begin{itemize}
    \item Set up lab. machine with python version of \textit{ATLAS-PROJECT}
    \item Completed \textbf{Exercise 6.1}
    \item Completed \textbf{Exercise 6.2/6.3}
    \item Changed machine from 3 to \textbf{17}
    \subitem Password: Atlasfp4217
    \item Explored Analysis.py and LabNotebook.ipynb
    
    \item Created google sheets in Ben's google drive (Uni/ATLAS/ATLAS) (https://docs.google.com/spreadsheets/d/1ZvxdJ9FcFHIEeIkDnRsDUL7tZVwtQ7-E249czEe7vmM/edit#gid=0)
\end{itemize}

\subsubsection*{09:00}
Create shared overleaf document to share plots and lab notes:
\\
\textit{https://www.overleaf.com/project/60251ee135265fbc62f06417}
\\
\textbf{Exercise 6.1:}
\\
Derive a conversion from the kinematic variable set $(p_T, \phi, \eta)$ to $(p_x, p_y, p_z)$
\begin{itemize}
     
\item{Consider a spherical polar coordinate system ($r,\phi,\theta$):}
\begin{align}
    x = r\sin\theta\cos\phi\\
    y = r\sin\theta\sin\phi\\   z=r\cos\theta
\end{align}
\item{We can write the momentum components in this coordinate system:}
\begin{align}
    p_x = p\sin\theta\cos\phi \\
    p_y = p\sin\theta\sin\phi \\ p_z=p\cos\theta
\end{align}

\item{Now we can write this in terms of the kinematic variables $\phi$, transverse momentum and pseudorapidity:}
\begin{align}
  p_T = p\sin\theta \\
  \eta = -\ln{\left(\tan\left(\frac{\theta}{2}\right)\right)}
\end{align}

\item{$p_z$ and $p_y$ are simple:}
\begin{align}
    p_x = p_T\cos\phi\\
    p_y = p_T\sin\phi
\end{align}

\item{$p_z$ can be written by expanding  $\eta$ using the half tan identity and using the dependence of $p_z$ on $\cos\theta$}

\begin{align}
    \tan\left(\frac{\theta}{2}\right)= e^{-\eta}= \sqrt{\frac{1-\cos\theta}{1+\cos\theta}}\\
    \frac{1-\cos\theta}{1+\cos\theta} =    \frac{1-\frac{p_z}{p}}{1+\frac{p_z}{p}} =e^{-2\eta}
\end{align}

\item{Finally, by rearranging and noting that $p^2 = p_T^2 + p_z^2$,  we find:}

\begin{align}
    p_z = p_T\sinh\eta
\end{align}
\end{itemize}

\textbf{Exercise 6.2/6.3:}\\
Derive an expression for $m_{ll}$ for the decay $Z \rightarrow \mu^+ \mu^-$.

\begin{itemize}
\item{Invarient mass formula, where we let c=1:} 
\begin{align}
    m^2 = E^2 - |\mathbf{p}|^2  
\end{align}

\item{In our case, where the muon pair can be written as particles (1) and (2), we have: }
\begin{align}
   m_{ll}^2 = (E_1 +E_2)^2 - (|\mathbf{p}_1 +\mathbf{p}_2|)^2
\end{align}

\item{Expanding the above expression gives:}

\begin{align}
  m_{ll}^2 = (E_1^2+E_2^2+2E_1E_2) - (|\mathbf{p}_1|^2 +  |\mathbf{p}_2|^2 +  2\mathbf{p}_1 \cdot\mathbf{p}_2)
\end{align}

\item{By using the invariant mass equation (29), we can substitute in $m_{\mu}$ ($m_1$ and $m_2$).}

\begin{align}
   m_{ll}^2 = m_1^2 +m_2^2 + 2(E_1E_2 - \mathbf{p}_1\cdot\mathbf{p}_2)
\end{align}

\item{Noting the fact that these are ultrarelativistic particles, their rest mass becomes negligible, and so $E \approx p$. We can now write:}

\begin{align}
    m_{ll}^2 = 2(|\mathbf{p}_1||\mathbf{p}_2| - \mathbf{p}_1\cdot\mathbf{p}_2)
\end{align}

\item{Now we can expand this using the momenta in terms kinematic variables.}

\begin{multline}
    m_{ll}^2 = 2p_{T1}p_{T2}\left(\sqrt{\cos^2{\phi_1}+\sin^2{\phi_1}+\sinh^2{\eta_1}}\sqrt{\cos^2{\phi_2}+\sin^2{\phi_2}+\sinh^2{\eta_2}} -\\ \cos{\phi_1}\cos{\phi_2} - \sin{\phi_1}\sin{\phi_2} - \sinh{\eta_1}\sinh{\eta_2}\right)
\end{multline}
    

\item{By expanding and using both combined angle formulae and trigonometric relation identities, the above expression can be simplified to}

\begin{align}
   m_{ll} =  \sqrt{2p_{T1}p_{T2}\left(\cosh{(\eta_1-\eta_2)}- \cos{(\phi_1 - \phi_2)}\right)} 
\end{align}

\end{itemize}
\textit{http://opendata.atlas.cern/release/2020/documentation/datasets/dataset13.html}

\subsubsection*{\textbf{9:20} - Lead DG}
Logging onto manchine 3.

\subsubsection*{\textbf{9:30}}
Set up Jupyter notebook on lab machine 3, Password: Atlasfp423.

\subsubsection*{\textbf{10:00} - Lead BG}

\subsubsection*{\textbf{10:07}}
Issue in 5.2: access denied when specifying language

\subsubsection*{\textbf{10:24}}
Resolved: There was a temporary conflict on the machine

\subsubsection*{\textbf{11:42}}
Issue in 5.3 (3\&4): Token not provided, Tried on both computers- still no change.

\subsubsection*{\textbf{12:14}}
Issue in 5.3(7): Incorrect page in web browser (Only shows password).
\\
Resolved: New machine used (mu=3) (changed to mu=17).

ssh string:
\begin{lstlisting}
ssh -N -L localhost:1234:localhost:8888 atlaslab17@atlaslab17.blackett.manchester.
ac.uk
\end{lstlisting}

\subsubsection*{\textbf{14:00} - Lead: DG}
Exploring Analysis.py and LabNotebook.ipynb

\subsubsection*{\textbf{15:00} - Lead: BG}
Exploring Analysis.py and LabNotebook.ipynb

\subsubsection*{\textbf{16:00}}
Exploring Analysis.py and LabNotebook.ipynb

\subsubsection*{\textbf{17:00}}
Log off for the day.
