%! Author = ben
%! Date = 02/02/2021

% Preamble
\documentclass[11pt]{article}

% Packages
\usepackage[top=2.54cm, bottom=2cm, left=1.5cm, right=1.5cm]{geometry}
\usepackage{amsmath}

%New Commands
\newcommand{\unitr}{\underline{\hat{r}}}
\newcommand{\unit}[1]{\mathbf{\hat{#1}}}
\newcommand{\ul}[1]{\underline{#1}}
\newcommand{\pd}{\partial}
\newcommand{\vect}[1]{\mathbf{#1}}
\newcommand{\evec}[1]{\vect{e}_{#1}}
\newcommand{\<}{\guilsinglleft}
\renewcommand{\>}{\guilsinglright}
\renewcommand{\it}[1]{\textit{#1}}
\renewcommand{\bf}[1]{\textbf{#1}}
\newcommand{\Nabla}{\boldsymbol{\nabla}}
\newcommand{\lap}{\nabla^2}

\usepackage{hyperref}
\hypersetup{
    colorlinks=true,
    linkcolor=blue,
    filecolor=magenta,
    urlcolor=cyan,
}

% Document
\begin{document}

    This will be an empty chapter and I will put some text here

    \begin{equation}
        \label{eq:1}
        \sum_{i=0}^{\infty} a_i x^i
    \end{equation}

    The equation \ref{eq:1} shows a sum that is divergent. This formula
%    will later be used in the page \pageref{second}.

    For further references see \href{http://www.sharelatex.com}{Something
    Linky} or go to the next url: \url{http://www.sharelatex.com} or open
    the next file \href{run:./file.txt}{File.txt}

    It's also possible to link directly any word or
    \hyperlink{thesentence}{any sentence} in your document.

    Terry Wyatt is inviting you to a scheduled Zoom meeting at 9:00 on Tuesday 6th October.\\
    Topic: Terry Wyatt's ATLAS Lab. Experiment Zoom Meeting\\
    Join Zoom Meeting\\
    https://zoom.us/j/92221091854\\
    Meeting ID: 922 2109 1854\\
    Passcode: 061020\\
    %%%%%%%%%%%%%%%% Aims %%%%%%%%%%%%%%%%
    \section{Aims}
    \begin{itemize}
        \item To gain appreciation of the physics processes that can occur in high energy proton-proton collisions at the LHC
        \item To be introduced to event selection and measurement methods used in particle physics data analysis.
    \end{itemize}

    %%%%%%%%%%%%%%%% Objectives %%%%%%%%%%%%%%%%
    \section{Objectives}
    \begin{itemize}
        \item
        \item Compare real data collected by the ATLAS experiment with computer-simulated "Monte Carlo" data.
        \item Measure the cross sections for the production of the Z, W, and Higgs bosons at the LHC and estimate
        statistical and systematic uncertainties of the results.
        \item Gain experience of using \it{ROOT}
        \item Extend the event selections to other sources of events at the LHC, such as:
        \subitem Higgs decaying to two photons
        \subitem events containing tau leptons, pairs of vector bosons, top quarks
        \subitem potential sources of new physics from "Beyond the standard model".
    \end{itemize}

    %%%%%%%%%%%%%%%% Background Reading %%%%%%%%%%%%%%%%
    \section{Background Reading}

    \subsection{A femto-course in particle physics}

    \subsubsection{Leptons}
    Leptons = spin-$\frac{1}{2}$ fundamental particles\\
    Charges of $-1, 0, +1$\\
    Each generation has a flavour -  electron, muon, tau

    \subsubsection{Quarks}
    Quarks = spin-$\frac{1}{2}$ fundamental particles\\
    Up-type = charge $+\frac{2}{3}$\\
    Down-type = charge $-\frac{1}{3}$\\
    Bind together via the strong interaction to form Hadrons\\
    Baryons = 3 quarks\\
    Mesons = quark + anti-quark

    \subsubsection{Gauge Bosons}
    Responsible for mediating forces

    \subsubsection{Higgs Boson}
    Does NOT directly mediate a force.\\
    It's an excitation of the Higgs field\\
    Field gives rise to the masses of the Z \& W bosons. \\

    \subsubsection{Parton}
    Parton = any constituent of a hadron (q, $\bar{q}$, g)



    Main interaction of interest for this experiment = the production \& decay of Z, W and Higgs.\\
    Most easily identified decay:
    \begin{align}
        Z &\rightarrow l^+ l_-
        \\
        W^- &\rightarrow l^- \bar{\nu}_l
        \\
        H &\rightarrow ZZ^{(*)} \rightarrow llll
    \end{align}

    %%%%%%%%%%%%%%%% Getting Started %%%%%%%%%%%%%%%%
    \section{Getting Started}
    Start a Jupyter notebook server running on the lab machine.
    Can then connect to this server using a web browser from anywhere.

    \subsection{Logging into the Lab machine}
    \begin{enumerate}
        \item Open terminal
        \item \it{ssh atlaslab[lab machine \#]@atlaslab[lab machine \#].blackett.manchester.ac.uk}
        \item Answer yes
        \item Enter the password for the specific machine
    \end{enumerate}

    \subsection{Setting up the code - TODO once at the start of the experiment}\label{subsec:setting-up-the-code}
    A skeleton version of the code is providied.\\
    To work in python: \it{cp -r /opt/ATLAS-Project-py ATLAS-Project}\\
    Then type in:\\
    \begin{itemize}
        \item \it{cd ATLAS-Project}
        \item \it{mkdir outputPlots}
        \item \it{ls -l}
    \end{itemize}
    Can then log out using \it{exit}

    \subsection{Starting up the Jupyter notebook server}
    Need to be followed at the beginning of each lab day.\\
    Uses Singularity to run a Jupyter notebook server on the lab machine.
    \begin{enumerate}
        \item Log into the lab machine using~\ref{subsec:setting-up-the-code}

        \item Configure Singularity
        \\\\
        \textbf{export SINGULARITY\_TMPDIR=\$HOME\/.singularity/tmp} - sets up the scratch directory
        \\
        \textbf{singularity run -B /data/ATLAS:\$HOME/ATLAS-Project/ATLAS \$HOME/root\_6.22.00-conda.sif bash}
        \\\\
        Should see \textbf{INFO: Convert SIF file to sandbox...} and then after about a minute \textbf{Singularity$>$}

        \item Start the Jupyter Notebook
        \\\\
        \textbf{jupyter-notebook --no-browser --port=8888}
        
        \item Should expect text that looks like:
        \\\\
        \textit{Writing notebook server cookie secret to /home/atlaslab16/.local/share/jupyter/runtime/notebook cookie secret
        Loading IPython parallel extension
        Serving notebooks from local directory: /home/atlaslab16
        The Jupyter Notebook is running at: http://localhost:8888/?token=a7ea520d2f4d6f62ea2026ac0aa1edd0a48e935b03539bc4 or http://127.0.0.1:8888/?token=a7ea520d2f4d6f62ea2026ac0aa1edd0a48e935b03539bc4 Use Control-C to stop this server and shut down all kernels (twice to skip confirmation).
        .
        To access the notebook, open this file in a browser: file:///home/atlaslab16/.local/share/jupyter/runtime/nbserver-15700-open.html Or copy and paste one of these URLs: http://localhost:8888/?token=a7ea520d2f4d6f62ea2026ac0aa1edd0a48e935b03539bc4 or http://127.0.0.1:8888/?token=a7ea520d2f4d6f62ea2026ac0aa1edd0a48e935b03539bc4}
        \\\\
        Need to keep this terminal window open.
        
        \item Open another terminal window:
        \\\\
        \textbf{ssh -N -L localhost:1234:localhost:8888 atlaslab[MACHINE NUMBER]@atlaslab[MACHINE NUMBER].blackett.manchester.ac.uk}

        \item Enter the machine password  \\
        Also keep this terminal window open.

        \item Use any browser to connect to 
        \\\\
        \textbf{localhost:1234}

        \item Should be presented with a Jupyter page asking for a \textbf{token} to be entered at the top of the page in the \textbf{token or password} box.

        \item Use the token given in the terminal window from a couple of steps ago.
    \end{enumerate}

    \subsection{Closing the Jupyter Notebook and Logging out of the lab machine}
    \begin{enumerate}
        \item \textbf{ctrl+C} on both terminal windows.
        \item then enter
        \\\\
        \textbf{y} | Close down the Jupyter notebook server\\
        \textbf{exit} | Closes Singularity\\
        \textbf{exit} | Logs out from the lab machine.
    \end{enumerate}

    \subsection{Using Analysis.py}
    Add names and date to the top of \textbf{Analysis.py} in a comment.\\
    \textbf{lep\_n}: Int = Number of leptons identified in each event.\\
    \textbf{lep\_pt}: Vect = vector of the lepton momenta in the plane transverse (perpendicular) to the beam direction (the z coordinate points along the beam).\\
    All available data sets: \url{http://opendata.atlas.cern/release/2020/documentation/datasets/dataset13.html}\\\\
    Values of energies and momenta are given in units of \textbf{MeV}.

    \subsection{Running the Skeleton code and examining the produced kistograms}
    Page 13 (14) for Monte Carlo data sets with string code.  (String code = What is typed in Run-Analysis.py)\\
    Full list ccan be found in the file \textbf{backend/dataSets.py}\\
    \\
    Remember to select \textit{Save and Checkpoint} from \textit{File} drop-down.
    Can then \textit{Revert to Checkpoint} from \textit{File} drop-down.
    \\\\
    The file \textit{Analysis.py} contains a single function, \textit{Analyse}.
    It is called once per data set analysed.
    In this function, will:
    \begin{itemize}
        \item select events
        \item plot histograms
        \item set histogram styles
        \item write plotted histograms to an output file.
    \end{itemize}

    %%%%%%%%%%%%%%%% Analysis %%%%%%%%%%%%%%%%
    \section{Analysis}

    Selection cut to select a sample of events corresponding to the desired particular physics processes.\\
    There is a trade off  between  the conflicting aims of achieving as high as possible a selection efficiency for the signal and rejecting as much as possible of the "background" from other physics processed.\\
    It's more important to make estimates of the selection efficiency and the level of residual background.
    And to come up with credible statistical and systematic uncertainties on these estimates.
    \\\\
    Note that the generated particles are not always actually observed in  the ATLAS detector.\\
    Note that the events may contain additional particles (e.g. jets of hadrons produced by radiation from the incoming partons before the annihilation that produced the Z boson)

    \subsection{Analysing Z bosons}

    ATLAS uses a right-handed coordinate system.
    Origin at the nominal interaction point in the centre of the detector and the z-axis along the beam pipe.
    The x-axis points from the interaction point to the centre of the LHC ring.
    y-axis upwards.
    \\\\
    In the data, the kinematic variables given for each lepton are:
    \begin{itemize}
        \item transverse (perpendicular) momentum - $p_T$
        \item azimuthal angle $\phi$ between $p_T$ and the x-axis
        \item pseudorapidity $\eta$
    \end{itemize}
    Pseudorapitdity is a spatial coordinate describing the angle of a particle relative to the beam axis defined as
    \begin{align}
        \eta = -ln \left( \tan\left( \frac{\theta}{2} \right) \right)
    \end{align}
    where $theta$ = the polar angle\\
    When high (small angle) the parricle is usually along the beam exis (often lost)
    For massless particles, this is equivalent to the rapidity ($y$) defined as
    \begin{align}
        y = \frac{1}{2}  \ln \left( \frac{|\vec{p}| + p_L}{|\vec{p}| - p_L} \right)
    \end{align}
    where $p_L$ is the longitudinal momentum and $\vec{p}$ is  the 3-momentum of the particle.

%    TODO: Exercise: Derive a conversion from the kinematic  variable set $(p_T, \phi, \eta)$ to $p_x, p_y, p_z$

    \subsubsection{Excercise 6.2}
    Derive an expression for $m_{ll}$ for the decay $Z \rightarrow \mu^+ \mu^-$
    \\\\
    The invarient mass of the system can be expressed directly using....
    \begin{align}
        m_{ll}^2 = 2 p_{T1} p_{T2} \left( \cosh (\eta_1 - \eta_2) - \cos(\phi_1 - \phi_2) \right)
    \end{align}
    
    \subsection{Making event selction cuts}
    Selection criteria/cuts on each event - e.g. only interested in events with oppositely charged leptons of the same type/flavour.
    \\
    Can restrict the events plotted by using the $p_T$ of the leptons.
    In general, always want to use the  particles with the highest $p_T$.
    
    \subsubsection{Exercise 6.6}
%    TODO: Find the invariant mass of Z \rightarrow on the MC data set and the 2lep data set
    
    
    Leptons produced in the decay of Z and W bosons tend to be "isolated" from other particles produced in p-p collisions.
    \\
    Leptons from "background" processes (e.g. decay of b quarks) tend to be accompanied by a jet of other particles.
    
    \subsubsection{The Isolation Variables - ptcone30 and etcone20}
    \textit{ptcone30} (ntuple)  = a sum over the $p_T$ of all tracks contained within a cone of half-width 0.3 in $\delta R$ around the lepton direction
    \begin{align}
        \Delta R = \sqrt{\Delta \phi^2 0}
    \end{align}
    \textit{etcone20} (ntuple) = contains a sum over the $p_T$ cone


    \subsection{Event weights for Monte Carlo data sets}
    The used ATLAS datasets correspond to an "integrated luminosity" $\int L dt$ = $10.064 \text{fb}^{-1}$ ("inverse femtobarns").
    = a measure of the number of the p=p collisions in ATLAS.
    \\
    The Monte Carlo
    \\\\
    Integral corresponds to the sum of all weights for all the calls of \textit{Fill} for which the plotted value lies between the lower and upper bounds of the histogram.
    It is \textit{Intergral} that should be compared NOT \textit{Entries} (when comparing histograms of ATLAS and MC data).


    \subsection{Cross Sections, backgrounds, and efficiencies}
    Cross section $\sigma$ for a given process such as $Z \rightarrow ll$:
    \begin{align}
        \sigma (pp \rightarrow Z \rightarrow ll) &= \frac{N^{selected} -  N^{background}}{\epsilon \int L dt}
        \\\\
        N^{\text{selected}} &= \text{ total number of events in the ATLAS data that pass the final selection cuts}
        \\
        N^{\text{background}} &= \text{ estimate of the number of background events in the selected data sample}
        \\
        N^{selected} -  N^{background} &= \text{estimate of the number of "signal" events in the ATLAS data for the targeted physics process}
%        \\
%        \epsilon &= \text{ efficiency for selecting the "signal" events. "signal" = whichever physics process that is having its cross section being measured.\\
%        Can be estimated using MC for the target sample process. \\
%        The sum of weights for all signal MC  events that pass the selection cuts needs to be divided by  that corresponding to all generated events for the relevant sample.\\
%        Info provided by running \textit{TotExpected.py}}
    \end{align}
    To get the sum of weights for all generated $Z \rightarrow ll$ signal MC events:
    \begin{align}
        & \text{\textbf{python3.4 TotExpected.py}}
        \\
        & \text{\textbf{Zee}}
    \end{align}

    Fractional uncertainty on $\int L dt$ = 1.7\%

    Other systematic uncertainties may arise, for example, from:
    \begin{itemize}
        \item backgrounds not accounted for
        \item disagreements between simulation and real data
    \end{itemize}
    Possible catch all to estimate the size of such effects = to re-calculate the cross section having changed your event selection cuts

    \subsection{Re-discovering the Higgs boson}
    Simplest Higgs decay mode:
    \begin{align}
        H \rightarrow ZZ^{(*)} \rightarrow llll
    \end{align}
     - Look at events producing 4 leptons\\
     - 2 same-flavour, opposite-charge pairs.


    \subsection{Analysing W bosons}
\end{document}